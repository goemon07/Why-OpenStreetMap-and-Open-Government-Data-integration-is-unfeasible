\section{Conclusions}
In this paper, a complete analysis has been carried out regarding the licensing system adopted by OpenStreetMap, the guidelines adopted by the various public administrations that adhere to the Open Government Data movement, as well as the possible barriers (voluntarily erected or not) that legal concessions can create.
Despite the growing enthusiasm of recent years, which has come to create compared to the much-vaunted concept of OpenData, projects like OSM still struggle to leave the nest because of scary jurisdictions, which are claiming to be part of the OGD movement, but still struggle to fully embrace open politics in the reality.
Finally, a representative example has been taken into consideration: the municipality of Bologna, with its Open Data portal, represents the timid attempt to grant, with a clause on the current license incompatible with OSM, only a small dataset to the public domain compared to the multitude of data that could be lifeblood for the proliferation of new services and improvement of existing ones.