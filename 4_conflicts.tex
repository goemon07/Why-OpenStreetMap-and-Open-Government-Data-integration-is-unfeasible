\section{License conflicts: breaking OSM potential}
Despite a large amount of data in which OSM can count on thanks to the active community, the strength of the project can be maximised only when data is integrated by third-party data sources. Indeed, different sources have been used by the organization in order to double-check, enrich and improve already collected data.
OGD movement can represent a perfect subject to integrate inside OSM project, which can hypothetically lead to think to a single, open and public integrated platform at the service of all world-wide citizens.
Thus, try to integrate and different licenses applied on different sources is still a challenge for the LWG and the community itself, representing again the biggest boundary to overcome.

\subsection{Open Government Data}\label{previous}
In the last decade, the concept of \textit{Open Government Data} (OGD) is taking place in different countries. As explained and analyzed in \cite{ubaldi2013open}, OGD is a sort of movement with the aim of convincing different administrations (from the local one since an entire country) to develop and provide web portals and APIs for the benefit of all, to improve and create new services that can generate value for the citizen.

In \cite{jetzek2013generating}, 61 countries have been taken into account with the goal of finding a correlation between the welfare of a certain population and the ``openness" of the respective administration: the outcome underlines how OGD can effectively create value, from the health and wellbeing to the environmental, passing through education and even GDP monetary value.
Four major factors have been evaluated as the fundamental concepts that automatically feed the entire process of OGD, increasing more and more the created value:
\begin{itemize}
    \item \textit{Efficiency}, improving resource allocation which can be practically translated into improving administrative bureaucracy, cutting processing costs, sign agreements with other countries and so on;
    \item \textit{Innovation}, generated by the proliferation of new services on top of the shared data;
    \item \textit{Transparency}, cutting corruption and unfair competition between different parties;
    \item \textit{Participation}, creating a prolific field where to add experience and ideas to solve difficult social problems.
\end{itemize}

\subsection{Barriers of the integration}
Unfortunately, on top of all those good intentions, not all countries are yet able to grasp the potential.
Despite the multitude of references that praise the OGD movement, in \cite{janssen2012benefits} the fairy tale has been criticized, analyzing the scenario under a practical and disillusioned point of view. In the paper, authors underline how governments are still afraid of the new trend and even if some steps have been done, this is not enough to honor promises explained in \ref{previous}. Governments should just accept to lose some degrees of control over their data, otherwise what released is no more than mere and useless advertisement to sponsor how transparent and open an administration is. Open data need to be taken into account discussing the entire infrastructure, improving quality of datasets which, in most of the cases, are poor and pretty unusable, and developing into users and citizens a "sense of data", ensuring public engagement.

Moreover, also in this scene, licenses could represent a big barrier that can further drain the usability of shared data.
Indipendently from the kind of shared data, each administration decides also which license to apply on top of released information.
Most used license is, again, Creative Commons. Even \textit{European Data Portal}, the platform developed by the European Union with the aim of collecting all open data provided by the 28 European Member States and four EFTA countries\footnote{The European Free Trade Association (EFTA) is a regional trade organization and free commerce area composed by four European states: Iceland, Liechtenstein, Norway, and Switzerland.}, strongly suggest to provide public sector information under CC\_BY licenses\footnote{\href{https://www.europeandataportal.eu/elearning/en/module4/#/id/co-01}{European Data Portal} page related to licensing.}.

Beyond the described issues and criticisms over OGD movement, the idea of integrating such open datasets let imagine OSM as a great opportunity to maximize the potential of both involved parties.
On one hand, OSM would see an exponential increase in the amount of available data, potentially becoming the largest open database that goes so far beyond the simple geospatial service. On the other hand, the OGD movement would become extremely valuable in a context where developers (not only private individuals but also large corporations) could benefit more and more.
So, why was this integration never even considered? For sure, legal issues linked to licenses represents again one of the biggest obstacles to overcome.

LWG, together with CC\_BY, spent a lot of effort to try to find a way to handle licenses without explicitly ask for new permissions to different contributors to the involved data sources. Unfortunately, they fail to solve the problem, leaving the mandatory obligation to let the contributor waive over attribution and unrestricted distribution of data.
OSM Wiki provides an exhaustive table\footnote{OpenStreetMap Wiki, \href{https://wiki.openstreetmap.org/wiki/Import/ODbL_Compatibility}{ODbL compatibility table}.} in which all licenses are listed together with compatibility and, in case of collision, required waivers to been obtained. Confirming that table, in \cite{khayyat2015open} a similar study of incompatibility between ODbL and CC\_BY licenses has been confirmed in between a more extensive analysis, proposing CC0 as a good alternative for ODbL.
CC0 is fully compatible with OSM because of the complete rights waive in favor of the public domain. So actually, this license can potentially solve all license inconsistencies but is not used at all by governments. Furthermore, it's trivial to understand that administrations are not so keen to adopt a kind of license that grants a copyright waiver for all their data.

This \textit{deadlock state} blocks every possible efficient and profitable development. CC0 is the practical solution to feed both OSM and public added value growth, but governments are still reluctant to make the required divestiture.  

\subsection{Wrong approaches: Bologna's case of study}
Trying to analyze the types of licenses used by different cities that have decided to share datasets, Bologna can certainly be represented as an excellent case study.
The municipality decided to apply a CC\_BY 4.0 license over a multitude of datasets, including libraries, schools, cycle paths and so on. There are also a lot of 3d-models of the most famous and historical buildings and monuments of the city.

Unfortunately, as already explained, the used license is the reason why, looking for Bologna from OSM website, the city seems to be comparable to any other place that is not active at all for what concern open data. Since is a big and important city, point of interest are well indexed but is clearly noticeable that municipality open datasets are not involved in OSM.
But the most important and relevant point is the strange behavior that Bologna's administration decided to adopt: from the Bologna's Open Data portal there is a webpage dedicated to adopted licenses\footnote{Bologna's Open Data website, \href{http://dati.comune.bologna.it/node/3449}{CC BY 4.0 ODbl compliant}.}, in which is possible to find an addendum.
The clause is a waive in favor of OSM for what concern the ``\textit{Ortofoto Bologna}" dataset, in line with permissions that OSM ask for in its cover letter\footnote{\href{https://wiki.openstreetmap.org/wiki/CM/en/001_-_Cover_letter_and_waiver_template_for_CC_BY_v4}{OSM Cover email} to ask for necessary waives in case of CC\_BY 4.0 license.}.
Such an email template is provided to community's components to ask third party data owners to allow the incorporation of CC\_BY 4.0 data into OSM.

This is a clear bad example of ad hoc license, creating an impartial situation in which is difficult to find out a clear understanding of the meaning that stands behind that decision.
The impression gained during the study of these dynamics leads to think of a behavior reluctant to assume responsibility, derived from the uncertainty over multiple usages that the public domain can create with respect to released data.
Governments and administrations reserve the right, in so doing, to be blamed for hypothetical bad habits, avoiding also any mended interventions required to fix the scenario, overcoming the fact that this approach is a serious blow to the transparency and intrinsic ethics of open data.