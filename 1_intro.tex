\section*{Introduction}
Since the dawn of mankind, humans have felt the need to put on paper, as faithfully as possible, the distinctive features of every corner of the globe. That role was always assigned to men of culture, figures with a huge background who have spent a large amount of their life tracking edges of fields and traveling all around the world.
Analyzing the mapping process under a philosophical perspective, experts sometimes courageously claims that mapping is more interesting than the territory itself\cite{caquard2013cartographyI}.
More complex systems have been developed during the ages, always reserving a certain fascination for the most passionate to this practice.

Nowadays, the methods have been completely overturned. Thanks to technology, anyone can contribute to the enrichment of maps, integrating as much information as possible. This is the case of OpenStreetMap, a project that aims to collect and make available as much detailed and relevant information about places, roads, monuments and buildings.
But the potential does not stop at the simple mapping: the information collected can give rise to thousands of possible uses.
Since also small contributes need to be considered as relevant information, the concept of \textit{license} need to be taken into account.
In this paper, adopted licenses will be the main subject, as necessary as, often, problematic concerning the openness linked to the open data trend. This topic has been widely discussed as a tool to create value, but in real-life, a long way need to be covered to reach the maximum potential.

In the first chapter, OpenStreetMap ecosystem will be evaluated, focusing the attention on the process used to acquire data from the community. In the second section, licenses used by OSM in the past and today will be reviewed. Lastly, in the final chapter, an attempt of analysis has been produced in order to understand why and how licenses can still be the objective of discussion for what concern the integration of Open Government Data, by administrations, and OSM database, to create a centralized source of public and free geospatial database.